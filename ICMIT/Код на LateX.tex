\documentclass[12pt, a4paper]{article}

% --- PREAMBLE ---
\usepackage[utf8]{inputenc}
\usepackage[T2A]{fontenc}
\usepackage[ukrainian]{babel}
\usepackage{amsmath} % Для математичних формул
\usepackage{amssymb} % Для математичних символів
\usepackage[a4paper, margin=2cm]{geometry} % Налаштування полів
\usepackage{graphicx} % Для включення графіки (якщо потрібно)
\usepackage{array} % Для таблиць

% Налаштування відступів для абзаців
\setlength{\parindent}{1.25cm}
\setlength{\parskip}{0.5em} % Невеликий простір між абзацами

% --- DOCUMENT ---
\begin{document}

% --- TITLE PAGE (Based on Appendix A ) ---
\begin{titlepage}
    \centering
    \large
    МІНІСТЕРСТВО ОСВІТИ І НАУКИ УКРАЇНИ \\[2ex]
    КРЕМЕНЧУЦЬКИЙ НАЦІОНАЛЬНИЙ УНІВЕРСИТЕТ \\
    ІМЕНІ МИХАЙЛА ОСТРОГРАДСЬКОГО \\[4ex]
    Навчально-науковий інститут електричної інженерії та інформаційних технологій \\[2ex]
    Кафедра комп'ютерної інженерії та електроніки \\[10ex]
    
    \Large \textbf{ЗВІТ З ПРАКТИЧНИХ РОБІТ} \\[5ex]
    
    \normalsize з навчальної дисципліни \\
    \textbf{«ІМОВІРНІСНО-СТАТИСТИЧНІ МЕТОДИ ІНФОРМАЦИЙНИХ ТЕХНОЛОГІЙ»} \\[2ex]
    (Збірник прикладів розв'язування задач) \\[10ex]
    
    % Використовуємо \vfill для гнучкого розподілу простору
    \vfill 
    
    \begin{flushleft}
    \normalsize
    \textbf{Виконавець:} \\
    Студент гр. ________ \\
    ПІБ ________________ \\[4ex]
    
    \textbf{Викладач:} \\
    ПІБ ________________ \\
    \end{flushleft}
    
    \vfill
    
    \centering
    Кременчук 2024
\end{titlepage}

\newpage
\tableofcontents % Зміст
\newpage

\part*{Частина 1}
\addcontentsline{toc}{part}{Частина 1}

\section{Практична робота № 1: Елементи комбінаторики} [cite: 83]

\subsection{Приклади розв'язування задач}

\subsubsection{Приклад 1.1} [cite: 94]
В одного студента 5 книг, у іншого 9. Усі книги різні. Скількома способами студенти можуть провести обмін 1 книгу на 1 книгу?

\textbf{Розв'язання.} [cite: 96-98] Спочатку розглянемо, яким чином перший студент може обрати одну книгу з 5. Це можна зробити п'ятьма способами. Водночас другий студент може це зробити за допомогою дев'ятьма способів. Тоді скориставшись формулою (1.1) можна записати: $5 \times 9 = 45$. Тобто, студенти можуть провести обмін 1 книгу на 1 книгу 45 різними способами.

\subsubsection{Приклад 1.2} [cite: 115]
10 спортсменів розіграють одну золоту, одну срібну та одну бронзову медалі. Скількома способами ці медалі можуть бути розподілені між спортсменами.

\textbf{Розв'язання.} [cite: 117-118] Враховуючи, що медалі при розподілі не можуть повторюватися і порядок має значення, скористаємося формулою (1.2) i запишемо наступне: $A_{10}^{3} = 10 \cdot 9 \cdot 8 = 720$. Отже, медалі можуть бути розподілені між спортсменами 720 різними способами.

\subsubsection{Приклад 1.3} [cite: 124]
В одного студента 5 книг, у іншого 9. Усі книги різні. Скількома способами студенти можуть провести обмін 3 книги на 3 книги?

\textbf{Розв'язання.} [cite: 126-128] Враховуючи, що повторення неможливі, а порядок при відборі не має значення, скористаємося формулою (1.3) і запишемо кількість способів, якими перший студент може відібрати 3 книги з 5: $C_{5}^{3}$. Кількість способів, якими цю дію може зробити другий дорівнює відповідно $C_{9}^{3}$. Тоді за формулою (1.1) розв'язок задачі становитиме $C_{5}^{3} \cdot C_{9}^{3} = 10 \cdot 84 = 840$. Отже, студенти можуть провести обмін 3 книгами на 3 книги 840 різними способами.

\subsubsection{Приклад 1.4} [cite: 132]
Скільки тризначних чисел можливо створити з цифр 1, 2, 3, 4, 5, 6, 7 якщо кожна цифра може входити у число більше одного разу?

\textbf{Розв'язання.} [cite: 133-134] У цьому випадку цифри відбираються з поверненням, але з урахуванням порядку. Таким чином для розв'язку задачі можна скористатися формулою (1.4): $7^{3} = 7 \cdot 7 \cdot 7 = 343$.

\subsubsection{Приклад 1.5} [cite: 138]
Знайти число можливих результатів підкидання двох гральних кісток, якщо кістки вважаються нерозрізненими.

\textbf{Розв'язання.} [cite: 139-140] Враховуючи, що гральні кістки не розрізнюються і значення очок можуть дублюватися на обох кістках, розв'язок задачі можна звести до комбінаторної схеми вибору з повтореннями без урахування порядку і скористатися формулою (1.5). Таким чином отримаємо $C_{6+2-1}^{2} = C_{7}^{2} = 21$ можливий результат, який може випасти на двох кістках.

\newpage
\section{Практична робота № 2: Класичне визначення ймовірності} [cite: 189]

\subsection{Приклади розв'язування задач}

\subsubsection{Приклад 2.1} [cite: 213]
В урні 10 куль, з яких 3 білих і 7 чорних. Яка ймовірність того, що навмання витягнута куля з цієї урни виявиться білою?

\textbf{Розв'язання.} [cite: 215-216] Нехай подія А полягає в тому, що витягнута куля виявляється білою. Цей іспит має 10 рівноймовірних результатів, з який для події А є сприятливими три. Отже, $p(A) = \frac{3}{10}$.

\subsubsection{Приклад 2.2} [cite: 217]
3 5 літер абетки складено слово «книга». Дитина, що не вміє читати, розсипала букви цього слова випадково. Знайти ймовірність того, що буде знову отримано слово «книга».

\textbf{Розв'язання.} [cite: 219-222] Один з можливих способів розв'язанні задачі полягає у наступному. Нехай подія А полягає в тому, що ми знову отримаємо слово «книга». Кількість всіх можливих «слів», які можуть утворитися при падінні дорівнює $5!$, але тільки один шанс сприяє тому, що випаде слово «книга». Тоді згідно з класичним визначенням ймовірності: $p(A) = \frac{k}{n} = \frac{1}{5!}$.

\subsubsection{Приклад 2.3} [cite: 223]
В урні лежать 45 куль, серед яких 6 білих. Витягуються три кулі без повернення. Визначити ймовірність витягування 3 білих куль.

\textbf{Розв'язання.} [cite: 226-228] Позначимо шукану подію через А, а відповідну ймовірність через $p(A)$. Згідно з класичним визначенням ймовірності $p(A) = \frac{k}{n}$ де у контексті задачі $k$ - кількість елементарних подій, що сприяють події А, a $n$ - кількість усіх рівноможливих способів витягнути 6 куль з 45 без повернення і без урахування порядку. Значить, $n = C_{45}^{3}$. Водночас $k$ буде дорівнювати кількості всіх способів, яким можна вилучити 3 білі кулі з 6 білих «і» в комбінації з всіма можливими способами, якими можна відібрати 3 чорні кулі з 39 чорних: $k = C_{6}^{3} \cdot C_{42}^{3}$ (sic)[cite: 228]. Таким чином розв'язок задачі буде виглядати так: $p(A) = \frac{C_{6}^{3} \cdot C_{39}^{3}}{C_{45}^{6}} \approx 0,03484$.

\newpage
\section{Практична робота № 3: Геометрична ймовірність, Теореми...} [cite: 283]

\subsection{Приклади розв'язування задач}

\subsubsection{Приклад 3.1} [cite: 312]
Точку кинуто в коло радіуса R. Знайти ймовірність того, що вона влучить у площину вписаного квадрата.

\textbf{Розв'язання.} [cite: 316-318] Знайдемо площу круга та квадрата. Площа круга: $S_{\text{круга}} = \pi R^{2}$. Площа вписаного квадрата: $S_{\text{квадрата}} = 2R^{2}$. Тоді, згідно з (3.1), ймовірність того, що точка влучить у площину вписаного квадрата, дорівнює відношенню площі квадрата до площі круга: $P = \frac{S_{\text{квадрата}}}{S_{\text{круга}}} = \frac{2R^{2}}{\pi R^{2}} = \frac{2}{\pi}$.

\subsubsection{Приклад 3.2} [cite: 328]
В урні 7 білих і 3 чорних кульки. Навмання витягають дві кульки без повернення. Яка ймовірність того, що вони обидві виявилися білого кольору.

\textbf{Розв'язання.} [cite: 330] $p(AB) = p(A)p(B/A) = \frac{7}{10} \cdot \frac{6}{9}$.

\subsubsection{Приклад 3.3} [cite: 333]
Є коробка з 9 новими тенісними м'ячами. Для гри беруть 3 м'ячі і після гри кладуть їх назад у коробку. Різниці між м'ячами, що використовувалися у грі, і новими м'ячами немає. Знайти ймовірність того, що після 3 ігор в коробці не залишиться жодного м'яча, що не використовувався у грі.

\textbf{Розв'язання.} [cite: 337] $\frac{1 \cdot 1 \cdot 1}{p(A_1)} \frac{\frac{6}{9} \cdot \frac{5}{8} \cdot \frac{4}{7}}{p(A_2/A_1)p(A_3/A_1 A_2)}$.

\subsubsection{Приклад 3.4} [cite: 341]
Розв'язати задачу з прикладу 3.2 за умови, що кульки витягаються з поверненнями.

\textbf{Розв'язання.} [cite: 342] $p(AB) = p(A)p(B) = \frac{7}{10} \cdot \frac{7}{10}$.

\subsubsection{Приклад 3.5} [cite: 344]
Два стрілка зробили по одному пострілу по мішені. Ймовірність влучення в мішень для першого стрілка складає 0,6, для другого 0,7. Знайти ймовірність того, що:
\begin{itemize}
    \item[а)] тільки один стрілок влучить у мішень;
    \item[б)] хоча б один стрілок влучить у мішень;
    \item[в)] обидва стрілка влучать у мішень;
    \item[г)] жоден стрілок не влучить у мішень;
    \item[д)] хоча б один стрілок не влучить у мішень.
\end{itemize}

\textbf{Розв'язання.} [cite: 349-361] Позначимо ймовірність влучення для першого стрілка як $P(A) = 0.6$ і для другого стрілка як $P(B) = 0.7$.
\begin{itemize}
    \item[а)] $P(\text{тільки один}) = P(A \cap \overline{B}) + P(\overline{A} \cap B) = P(A)(1-P(B)) + (1-P(A))P(B) = 0.6 \cdot 0.3 + 0.4 \cdot 0.7 = 0.18 + 0.28 = 0.46$.
    \item[б)] $P(\text{хоча б один}) = 1 - P(\overline{A} \cap \overline{B}) = 1 - (1-P(A))(1-P(B)) = 1 - 0.4 \cdot 0.3 = 1 - 0.12 = 0.88$.
    \item[в)] $P(\text{обидва}) = P(A \cap B) = 0.6 \cdot 0.7 = 0.42$.
    \item[г)] $P(\text{жоден}) = P(\overline{A} \cap \overline{B}) = (1-P(A))(1-P(B)) = 0.4 \cdot 0.3 = 0.12$.
    \item[д)] $P(\text{хоча б один не влучить}) = 1 - P(A \cap B) = 1 - 0.6 \cdot 0.7 = 1 - 0.42 = 0.58$.
\end{itemize}

\subsubsection{Приклад 3.6} [cite: 364]
Є 3 заводи... 1-й завод виробляє 25\%, 2-й завод 35\%, 3-й завод 40\%... Брак складає 5\% від продукції 1-го заводу, 3\% від 2-го, 4\% від 3-го... Знайти:
\begin{itemize}
    \item[а)] ймовірність покупки бракованого кристалу;
    \item[б)] умовну ймовірність того, що куплений виріб виготовлено 1-им заводом, якщо цей кристал виявився бракованим.
\end{itemize}

\textbf{Розв'язання.} [cite: 370-372]
\begin{itemize}
    \item[а)] $P(A) = 0.05 \cdot 0.25 + 0.03 \cdot 0.35 + 0.04 \cdot 0.4 = 0.0125 + 0.0105 + 0.016 = 0.039$.
    \item[б)] $P(H_1|A) = \frac{P(H_1)P(A|H_1)}{P(A)} = \frac{0.05 \cdot 0.25}{0.05 \cdot 0.25 + 0.03 \cdot 0.35 + 0.04 \cdot 0.4} = \frac{0.0125}{0.039} \approx 0.32$.
\end{itemize}

\subsubsection{Приклад 3.7} [cite: 388]
Повернемося до попереднього прикладу. Розглянемо три гіпотези:
$H_1$ - 1-й з.; $P(H_1) = 0.25$ [cite: 390]
$H_2$ - 2-й з.; $P(H_2) = 0.35$ [cite: 391]
$H_3$ - 3-й з.; $P(H_3) = 0.4$ [cite: 393]
Нехай $A = \{\text{виріб виявився бракованим}\}$, тоді:
$P(A|H_1) = 0.05$ [cite: 396]
$P(A|H_2) = 0.03$ [cite: 397]
$P(A|H_3) = 0.04$ [cite: 398]

\newpage
\section{Практична робота № 4: Схема Бернуллі} [cite: 489]

\subsection{Приклади розв'язування задач}

\subsubsection{Приклад 4.1} [cite: 502]
Монету кинуто $n=3$ рази. Яка ймовірність того, що орел випаде рівно $k=1$ раз?

\textbf{Розв'язання.} [cite: 504-509]
\textbf{І спосіб.} $p(A) = p(A_1 \overline{A}_2 \overline{A}_3 + \overline{A}_1 A_2 \overline{A}_3 + \overline{A}_1 \overline{A}_2 A_3) = p(A_1 \overline{A}_2 \overline{A}_3) + p(\overline{A}_1 A_2 \overline{A}_3) + p(\overline{A}_1 \overline{A}_2 A_3) = p(A_1)p(\overline{A}_2)p(\overline{A}_3) + p(\overline{A}_1)p(A_2)p(\overline{A}_3) + p(\overline{A}_1)p(\overline{A}_2)p(A_3) = 3 \cdot \frac{1}{2} \cdot \frac{1}{2} \cdot \frac{1}{2} = \frac{3}{8}$.
\textbf{ІІ спосіб.} Скористаємося формулою Бернуллі:
$p_n(k) = C_{n}^{k} p^k q^{n-k}$
$p_3(1) = C_{3}^{1} p^1 q^{3-1} = \frac{3!}{1!(3-1)!} (0.5)^1 (1-0.5)^2 = 3 \cdot 0.5 \cdot (0.5)^2 = 3 \cdot (0.5)^3 = \frac{3}{8}$.

\subsubsection{Приклад 4.2} [cite: 526]
Яка ймовірність того, що при $n=1000$ киданнях монети орел випаде рівно $k=500$ разів?

\textbf{Розв'язання.} [cite: 528-533] (Тут $n=100$ у розв'язку [cite: 529])
Так як $npq = 100 \cdot 0.5 \cdot 0.5 = 25 > 10$, то доцільно скористатися наближеною формулою Лапласа:
$p_n(k) \approx \frac{1}{\sqrt{npq}} \phi(x)$.
$x = \frac{k - np}{\sqrt{npq}} = \frac{500 - 1000 \cdot 0.5}{\sqrt{25}} = 0$. (Тут $n=1000$ [cite: 531])
Таким чином маємо: $p_{1000}(500) = \frac{1}{5} \phi(0) = \frac{1}{5} \cdot \frac{1}{\sqrt{2\pi}} \approx 0.03 \approx 3\%$ (sic)[cite: 533].

\subsubsection{Приклад 4.3} [cite: 546]
Імовірність настання події А в кожному з 900 незалежних дослідів дорівнює $p=0.8$. Визначте імовірність того, що подія А відбудеться: а) 750 разів; б) 710 разів; в) від 710 до 740 разів.

\textbf{Розв'язання.} [cite: 548] (Використовуємо $\sqrt{npq} = \sqrt{144}=12$, попри описку "14,4" [cite: 548])
$n=900, p=0.8, q=0.2, np=720, \sqrt{npq} = 12$.
\begin{itemize}
    \item[а)] $x = \frac{750 - 720}{12} = 2.5$; $\phi(2.5) \approx 0.0175$ [cite: 549]
    $P_{900}(750) \approx \frac{1}{12} \phi(2.5) = \frac{1}{12} \cdot 0.0175 \approx 0.00146$ [cite: 550]
    
    \item[б)] $x = \frac{710 - 720}{12} \approx -0.83$; $\phi(-0.83) = \phi(0.83) \approx 0.2827$ [cite: 551]
    $P_{900}(710) \approx \frac{1}{12} \cdot 0.2827 \approx 0.0236$ [cite: 552]
    
    \item[в)] $x_1 = \frac{710 - 720}{12} \approx -0.83$; $x_2 = \frac{740 - 720}{12} \approx 1.67$ [cite: 553]
    $\Phi(-0.83) \approx -0.2967$; $\Phi(1.67) \approx 0.4525$ [cite: 554]
    $P_{900}(710 \le k \le 740) \approx \Phi(x_2) - \Phi(x_1) = \Phi(1.67) - \Phi(-0.83) = \Phi(1.67) + \Phi(0.83)$
    $P \approx 0.4525 + 0.2967 = 0.7492$ [cite: 555]
\end{itemize}

\subsubsection{Приклад 4.4} [cite: 565]
Імовірність того, що електролампочка... бракованою, дорівнює 0,02. ...відібрано 1000 лампочок. Оцініть ймовірність того, що частота... відрізняється від 0,02 менш ніж на 0,01.

\textbf{Розв'язання.} [cite: 568-574] $p=0.02, n=1000, \epsilon=0.01$. $P(|\frac{k}{1000} - 0.02| < 0.01)$.
$P(|\frac{k}{n} - p| \le \epsilon) \approx 2\Phi(\epsilon \sqrt{\frac{n}{pq}})$.
$npq = 1000 \cdot 0.02 \cdot 0.98 = 19.6 > 10$. $\sqrt{npq} \approx 4.43$.
$x_1 = \frac{11 - 1000 \cdot 0.02}{\sqrt{19.6}} \approx -2.03$; $x_2 = \frac{29 - 20}{4.43} \approx 2.03$ [cite: 572]
$\Phi(-2.03) \approx -0.4788$; $\Phi(2.03) \approx 0.4788$ [cite: 573]
$P_{1000}(11 \le k \le 29) \approx \Phi(2.03) - \Phi(-2.03) = 0.4788 + 0.4788 = 0.9576$.

\subsubsection{Приклад 4.5} [cite: 576]
... $p=0.1$. ... $n=400$ деталей. ...відносна частота... відхилиться... не більше ніж на $0.03$.

\textbf{Розв'язання.} [cite: 577-578] $n=400, p=0.1, q=0.9, \epsilon=0.03$.
$P(|m/400 - 0.1| \le 0.03) \approx 2\Phi(\epsilon \sqrt{\frac{n}{pq}})$
$P \approx 2\Phi(0.03 \sqrt{\frac{400}{0.1 \cdot 0.9}}) = 2\Phi(0.03 \sqrt{\frac{400}{0.09}}) = 2\Phi(0.03 \cdot \frac{20}{0.3}) = 2\Phi(2) = 2 \cdot 0.4772 = 0.9544$.

\subsubsection{Приклад 4.6} [cite: 588]
... 5000 доброякісних виробів, $p=0.0002$ - ймовірність... пошкодження... Знайти ймовірність... рівно 3 пошкоджених вироби.

\textbf{Розв'язання.} [cite: 590-591] $\lambda = np = 5000 \cdot 0.0002 = 1$.
$P_n(k) \approx \frac{\lambda^k e^{-\lambda}}{k!}$
$p_{5000}(3) = \frac{1^3}{3!} e^{-1} = \frac{1}{6e} \approx 0.06$.

\subsubsection{Приклад 4.7} [cite: 593]
Телефонна станція обслуговує 400 абонентів. $p=0.01$...
а) 5 абонентів; б) не більш 4; в) не менш 3.

\textbf{Розв'язання.} [cite: 596] $\lambda = np = 400 \cdot 0.01 = 4$. $e^{-4} \approx 0.018316$.
\begin{itemize}
    \item[а)] $P_{400}(5) \approx \frac{4^5}{5!} e^{-4} = \frac{1024}{120} e^{-4} \approx 0.156293$ [cite: 597]
    \item[б)] $P_{400}(0 \le k \le 4) \approx P(0)+P(1)+P(2)+P(3)+P(4)$ [cite: 599-600]
    $P \approx 0.018316 + 0.073263 + 0.146525 + 0.195367 + 0.195367 = 0.628838$
    \item[в)] $P_{400}(k \ge 3) = 1 - P_{400}(0 \le k \le 2) = 1 - (P(0)+P(1)+P(2))$ [cite: 601-602]
    $P \approx 1 - 0.018316 - 0.073263 - 0.146525 = 0.761896$
\end{itemize}

\newpage
\part*{Частина 2}
\addcontentsline{toc}{part}{Частина 2}

\section{Практична робота № 1: Закони розподілу та числові характеристики} [cite: 903]

\subsection{Приклади розв'язування задач}

\subsubsection{Приклад 1.1} [cite: 911]
Зіставимо двом сторонам монети ДВВ відповідним чином: «Орел» - 1, «Решка» - 0. ДВВ Х - випадіння нуля чи одиниці в одному киданні. Записати закон розподілу ДВВ у табличному вигляді.

\textbf{Розв'язання.} [cite: 913]
\begin{center}
\begin{tabular}{|c|c|c|}
\hline
$X$ & 0 & 1 \\
\hline
$p$ & $\frac{1}{2}$ & $\frac{1}{2}$ \\
\hline
\end{tabular}
\end{center}
$\sum p_i = p_1 + p_2 = \frac{1}{2} + \frac{1}{2} = 1$[cite: 915].

\subsubsection{Приклад 1.2} [cite: 926]
Знайти математичне сподівання ДВВ, заданою умовою задачі прикладу 1.1.

\textbf{Розв'язання.} [cite: 927-928] Відповідно до визначення МС запишемо:
$M(x) = \sum_{i=1}^{n} x_i p_i = 0 \cdot \frac{1}{2} + 1 \cdot \frac{1}{2} = \frac{1}{2}$.

\subsubsection{Приклад 1.3} [cite: 936]
Знайти дисперсію ДВВ, заданою умовою задачі прикладу 1.1.

\textbf{Розв'язання.} [cite: 938-939] Згідно з визначенням дисперсії ДВВ запишемо:
$D(x) = M(x^2) - [M(x)]^2 = (0^2 \cdot \frac{1}{2} + 1^2 \cdot \frac{1}{2}) - \left[ \frac{1}{2} \right]^2 = \frac{1}{2} - \frac{1}{4} = \frac{1}{4}$.

\subsubsection{Приклад 1.4} [cite: 955]
Знайти функцію розподілу ДВВ, заданою умовою задачі прикладу 1.1 і побудувати її графік.

\textbf{Розв'язання.} [cite: 958]
$$ F(X) = p(x < X) = 
\begin{cases} 
    0, & x \le 0 \\
    \frac{1}{2}, & 0 < x \le 1 \\
    1, & x > 1 
\end{cases} $$

\subsubsection{Приклад 1.5} [cite: 973]
Говорять, що Х має рівномірний розподіл на відрізку [a, b] ... $f(x) = \frac{1}{b-a}$ для $a < x \le b$.
Знайти $F(x)$, $M(X)$ та $D(X)$.

\textbf{Розв'язання.} [cite: 980-985]
Згідно з визначенням $F(x) = \int_{-\infty}^{x} f(t)dt$.
$$ F(x) = 
\begin{cases} 
    0, & x \le a \\
    \frac{x-a}{b-a}, & a < x \le b \\
    1, & x > b 
\end{cases} $$
(Виведення $M(X)$ та $D(X)$ у прикладі відсутнє [cite: 973-991], наводиться лише $F(x)$).

\subsubsection{Приклад 1.6} [cite: 1022]
НВВ Х має нормальний розподіл з $\mu=3$ та $\sigma=1$.
Обчислити $P(1 \le X \le 2)$ та $P(|X-\mu| \le \delta)$ для $\delta=0.01$.

\textbf{Розв'язання.} [cite: 1024-1030]
$p(1 \le X \le 2) = \Phi\left(\frac{2-3}{1}\right) - \Phi\left(\frac{1-3}{1}\right) = \Phi(-1) - \Phi(-2)$
Оскільки $\Phi(-x) = -\Phi(x)$, то:
$p = - \Phi(1) - (-\Phi(2)) = \Phi(2) - \Phi(1) = 0.4772 - 0.3413 = 0.1359$.
Для $P(|X-3| \le 0.01)$, скористаємося $P(|X-\mu| < \delta) = 2\Phi(\frac{\delta}{\sigma})$:
$P(|X-3| < 0.01) = 2\Phi\left(\frac{0.01}{1}\right) = 2\Phi(0.01) \approx 2 \cdot 0.004 = 0.008$.
(У методичці наведено $2 \cdot 0.000 = 0.000$[cite: 1030], що є грубим округленням).

\newpage
\section{Практична робота № 2: Закони розподілу функцій} [cite: 1116]

\subsection{Приклади розв'язування задач}

\subsubsection{Приклад 2.1} [cite: 1136]
Випадкова величина Х розподілена за нормальним законом з $\mu=0$. Знайти закон розподілу $Y=X^3$.

\textbf{Розв'язання.} [cite: 1137-1138]
$f(x) = \frac{1}{\sigma\sqrt{2\pi}} e^{-\frac{x^2}{2\sigma^2}}$.
Функція $y=x^3$ монотонно зростає. $y' = 3x^2$.
$x = \sqrt[3]{y} = y^{1/3}$. $\psi(y) = y^{1/3}$. $\psi'(y) = \frac{1}{3}y^{-2/3}$.
$g(y) = f(\psi(y)) \cdot |\psi'(y)|$
$g(y) = \frac{1}{\sigma\sqrt{2\pi}} e^{-\frac{(y^{1/3})^2}{2\sigma^2}} \cdot \left| \frac{1}{3}y^{-2/3} \right| = \frac{1}{3\sigma\sqrt{2\pi}} e^{-\frac{y^{2/3}}{2\sigma^2}} y^{-2/3}$.
(Примітка: у методичці [cite: 1138] є помилка в знаменнику $\sigma\sqrt{2\pi}y^{2/3}$ замість $\sigma\sqrt{2\pi} \cdot 3y^{2/3}$ та $e^{\frac{...}{|(y^{1/3})'|}}$, що невірно).

\subsubsection{Приклад 2.2} [cite: 1148]
Дві незалежні випадкові величини $\xi$ і $\eta$ мають стандартний нормальний розподіл: $\xi \sim N(0,1)$, $\eta \sim N(0,1)$. Покажемо, що $\xi + \eta \sim N(0,2)$.

\textbf{Розв'язання.} [cite: 1150-1153]
$f_{\xi+\eta}(x) = \int_{-\infty}^{\infty} f_{\xi}(u) \cdot f_{\eta}(x-u) du = \int_{-\infty}^{\infty} \frac{1}{\sqrt{2\pi}} e^{-\frac{1}{2}u^2} \cdot \frac{1}{\sqrt{2\pi}} e^{-\frac{1}{2}(x-u)^2} du$
$f_{\xi+\eta}(x) = \frac{1}{2\pi} \int_{-\infty}^{\infty} e^{-\frac{1}{2}(u^2 + (x-u)^2)} du = \frac{1}{2\pi} \int_{-\infty}^{\infty} e^{-\frac{1}{2}(2u^2 - 2xu + x^2)} du$
$f_{\xi+\eta}(x) = \frac{1}{2\pi} \int_{-\infty}^{\infty} e^{-\left(u^2 - xu + \frac{x^2}{2}\right)} du$
Виділимо повний квадрат у показнику: $u^2 - xu + \frac{x^2}{2} = (u - \frac{x}{2})^2 - \frac{x^2}{4} + \frac{x^2}{2} = (u - \frac{x}{2})^2 + \frac{x^2}{4}$.
$f_{\xi+\eta}(x) = \frac{1}{2\pi} \int_{-\infty}^{\infty} e^{-\left((u - \frac{x}{2})^2 + \frac{x^2}{4}\right)} du = \frac{1}{2\pi} e^{-\frac{x^2}{4}} \int_{-\infty}^{\infty} e^{-(u - \frac{x}{2})^2} du$
Заміна $v = u - \frac{x}{2}$, $dv = du$.
$\int_{-\infty}^{\infty} e^{-v^2} dv = \sqrt{\pi}$ (інтеграл Пуассона).
$f_{\xi+\eta}(x) = \frac{1}{2\pi} e^{-\frac{x^2}{4}} \cdot \sqrt{\pi} = \frac{1}{2\sqrt{\pi}} e^{-\frac{x^2}{4}}$
Це $N(0, 2)$, оскільки $\frac{1}{\sqrt{2\pi \sigma^2}} e^{-\frac{(x-\mu)^2}{2\sigma^2}}$.
$\mu=0, \sigma^2=2$. $\frac{1}{\sqrt{2\pi \cdot 2}} e^{-\frac{x^2}{2 \cdot 2}} = \frac{1}{\sqrt{4\pi}} e^{-\frac{x^2}{4}} = \frac{1}{2\sqrt{\pi}} e^{-\frac{x^2}{4}}$.

\end{document}
\documentclass[12pt, a4paper]{article}

% --- Пакети для української мови та математики ---
\usepackage[utf8]{inputenc}
\usepackage[T2A]{fontenc}
\usepackage[ukrainian]{babel}
\usepackage{amsmath, amssymb, amsfonts}
\usepackage{geometry}
\usepackage{graphicx}
\usepackage{indentfirst}
\usepackage{titlesec}
\usepackage{tocloft}

% --- Налаштування сторінки ---
\geometry{
    left=2.5cm,
    right=1.5cm,
    top=2cm,
    bottom=2cm
}

% --- Налаштування змісту ---
\renewcommand{\cftsecleader}{\cftdotfill{\cftdotsep}}

% --- Дані студента (зі зразка) ---
\newcommand{\studentGroup}{КН-24-1}
\newcommand{\studentName}{Дон А.А.}
\newcommand{\teacherName}{Сидоренко В.М.}

\begin{document}

% ================= ТИТУЛЬНА СТОРІНКА =================
\begin{titlepage}
    \centering
    \large
    МІНІСТЕРСТВО ОСВІТИ І НАУКИ УКРАЇНИ \\[0.5cm]
    КРЕМЕНЧУЦЬКИЙ НАЦІОНАЛЬНИЙ УНІВЕРСИТЕТ \\
    ІМЕНІ МИХАЙЛА ОСТРОГРАДСЬКОГО \\[1cm]
    
    \normalsize
    Навчально-науковий інститут електричної інженерії та інформаційних технологій \\
    Кафедра комп'ютерної інженерії та електроніки \\[3cm]
    
    \Large \textbf{ЗВІТ З ПРАКТИЧНИХ РОБІТ} \\[1cm]
    \large з навчальної дисципліни \\
    \textbf{«ІМОВІРНІСНО-СТАТИСТИЧНІ МЕТОДИ ІНФОРМАЦІЙНИХ ТЕХНОЛОГІЙ»} \\
    (Збірник прикладів розв'язування задач) \\[4cm]
    
    \begin{flushright}
        \begin{minipage}{0.45\textwidth}
            \large
            \textbf{Виконавець:} \\
            Студент гр. \studentGroup \\
            \studentName \\[1cm]
            
            \textbf{Викладач:} \\
            \teacherName
        \end{minipage}
    \end{flushright}
    
    \vfill
    
    Кременчук 2024
\end{titlepage}

% ================= ЗМІСТ =================
\tableofcontents
\newpage

% ================= ЧАСТИНА 1 =================
\part*{Частина 1}
\addcontentsline{toc}{part}{Частина 1}

\section{Практична робота № 1: Елементи комбінаторики}
\subsection{Приклади розв'язування задач}

\subsubsection{Приклад 1.1}
В одного студента 5 книг, у іншого 9. Усі книги різні. Скількома способами студенти можуть провести обмін 1 книгу на 1 книгу?

\textbf{Розв'язання:}
Спочатку розглянемо, яким чином перший студент може обрати одну книгу з 5. Це можна зробити п'ятьма способами. Водночас другий студент може це зробити за допомогою дев'ятьма способів. Тоді скориставшись формулою перемноження шансів можна записати:
$$ N = 5 \times 9 = 45 $$
[cite_start]Тобто, студенти можуть провести обмін 1 книгу на 1 книгу 45 різними способами[cite: 96].

\subsubsection{Приклад 1.2}
10 спортсменів розіграють одну золоту, одну срібну та одну бронзову медалі. Скількома способами ці медалі можуть бути розподілені?

\textbf{Розв'язання:}
Враховуючи, що медалі при розподілі не можуть повторюватися і порядок має значення, скористаємося формулою розміщень:
$$ A_{10}^{3} = 10 \cdot 9 \cdot 8 = 720 $$
[cite_start]Отже, медалі можуть бути розподілені між спортсменами 720 різними способами[cite: 117].

\subsubsection{Приклад 1.3}
В одного студента 5 книг, у іншого 9. Скількома способами студенти можуть провести обмін 3 книги на 3 книги?

\textbf{Розв'язання:}
Порядок не має значення. Перший студент може відібрати 3 книги з 5 способами $C_{5}^{3}$. Другий — $C_{9}^{3}$. Загальна кількість:
$$ N = C_{5}^{3} \cdot C_{9}^{3} = 10 \cdot 84 = 840 $$
[cite_start][cite: 126].

\subsubsection{Приклад 1.4}
Скільки тризначних чисел можливо створити з цифр 1..7, якщо цифри можуть повторюватися?

\textbf{Розв'язання:}
Вибірка з поверненням з урахуванням порядку:
$$ N = n^k = 7^3 = 343 $$
[cite_start][cite: 133].

\subsubsection{Приклад 1.5}
Знайти число можливих результатів підкидання двох гральних кісток, якщо кістки нерозрізнені.

\textbf{Розв'язання:}
Комбінації з повтореннями:
$$ C_{n+k-1}^{k} = C_{6+2-1}^{2} = C_{7}^{2} = 21 $$
[cite_start][cite: 139].

\section{Практична робота № 2: Класичне визначення ймовірності}
\subsection{Приклади розв'язування задач}

\subsubsection{Приклад 2.1}
В урні 10 куль (3 білі, 7 чорних). Яка ймовірність витягнути білу?

\textbf{Розв'язання:}
$$ p(A) = \frac{k}{n} = \frac{3}{10} = 0.3 $$
[cite_start][cite: 215].

\subsubsection{Приклад 2.2}
З літер слова «книга» розсипали букви. Яка ймовірність зібрати слово «книга» знову?

\textbf{Розв'язання:}
Загальна кількість перестановок $n = 5! = 120$. Сприятлива подія одна.
$$ p(A) = \frac{1}{120} $$
[cite_start][cite: 219].

\subsubsection{Приклад 2.3}
В урні 45 куль (6 білих). Витягують 3 кулі. Ймовірність, що всі 3 білі.

\textbf{Розв'язання:}
$$ p(A) = \frac{C_6^3 \cdot C_{39}^0}{C_{45}^3} \approx 0.03484 $$
[cite_start][cite: 226].

\section{Практична робота № 3: Геометрична ймовірність, Теореми}
\subsection{Приклади розв'язування задач}

\subsubsection{Приклад 3.1}
Ймовірність влучення точки в квадрат, вписаний у коло.

\textbf{Розв'язання:}
$$ P = \frac{S_{kv}}{S_{kr}} = \frac{2R^2}{\pi R^2} = \frac{2}{\pi} $$
[cite_start][cite: 316].

\subsubsection{Приклад 3.2}
Урна (7 білих, 3 чорних). Витягують 2 без повернення. Ймовірність, що обидві білі.

\textbf{Розв'язання:}
$$ p(AB) = p(A)p(B/A) = \frac{7}{10} \cdot \frac{6}{9} = \frac{42}{90} \approx 0.47 $$
[cite_start][cite: 330].

\subsubsection{Приклад 3.3}
Коробка з 9 м'ячами. 3 гри по 3 м'ячі (з поверненням). Ймовірність, що всі м'ячі побували у грі.

\textbf{Розв'язання:}
Розраховується за теоремою добутку для великої кількості подій:
$$ P = 1 \cdot \frac{C_6^3}{C_9^3} \dots $$
[cite_start](деталі у методичці [cite: 337]).

\subsubsection{Приклад 3.4}
Як приклад 3.2, але з поверненням.

\textbf{Розв'язання:}
$$ p(AB) = \frac{7}{10} \cdot \frac{7}{10} = 0.49 $$
[cite_start][cite: 342].

\subsubsection{Приклад 3.5}
Два стрілка (0.6 та 0.7).
\textbf{Розв'язання:}
а) Тільки один: $0.6 \cdot 0.3 + 0.4 \cdot 0.7 = 0.46$.
б) Хоча б один: $1 - 0.4 \cdot 0.3 = 0.88$.
в) Обидва: $0.42$.
[cite_start][cite: 349].

\subsubsection{Приклад 3.6}
Формула повної ймовірності (3 заводи).
\textbf{Розв'язання:}
$$ P(A) = 0.05 \cdot 0.25 + 0.03 \cdot 0.35 + 0.04 \cdot 0.4 = 0.039 $$
[cite_start][cite: 370].

\subsubsection{Приклад 3.7}
Формула Байєса для прикладу 3.6.
\textbf{Розв'язання:}
$$ P(H_1|A) = \frac{0.05 \cdot 0.25}{0.039} \approx 0.32 $$
[cite_start][cite: 388].

\section{Практична робота № 4: Схема Бернуллі}
\subsection{Приклади розв'язування задач}

\subsubsection{Приклад 4.1}
Монету кинуто 3 рази. Орел випаде 1 раз.
\textbf{Розв'язання:}
$$ P_3(1) = C_3^1 (0.5)^1 (0.5)^2 = \frac{3}{8} $$
[cite_start][cite: 504].

\subsubsection{Приклад 4.2}
$n=1000, k=500$. Локальна теорема Лапласа.
\textbf{Розв'язання:}
$$ P_{1000}(500) \approx \frac{1}{\sqrt{25}} \phi(0) \approx 0.03 $$
[cite_start][cite: 531].

\subsubsection{Приклад 4.3}
$n=900, p=0.8$. Інтегральна теорема.
\textbf{Розв'язання:}
[cite_start]Розрахунок ймовірності від 710 до 740 разів через функцію Лапласа $\Phi(x)$[cite: 548].

\subsubsection{Приклад 4.4}
Відхилення частоти (лампочки).
\textbf{Розв'язання:}
$$ P \approx 2\Phi(2.03) \approx 0.9576 $$
[cite_start][cite: 574].

\subsubsection{Приклад 4.5}
Відхилення частоти (деталі).
\textbf{Розв'язання:}
$$ P \approx 2\Phi(2) \approx 0.9544 $$
[cite_start][cite: 578].

\subsubsection{Приклад 4.6}
Формула Пуассона (завод).
\textbf{Розв'язання:}
$$ P_{5000}(3) \approx \frac{1^3 e^{-1}}{3!} \approx 0.06 $$
[cite_start][cite: 591].

\subsubsection{Приклад 4.7}
Телефонна станція (Пуассон).
\textbf{Розв'язання:}
[cite_start]Розрахунок для $\lambda=4$[cite: 596].

\newpage
% ================= ЧАСТИНА 2 =================
\part*{Частина 2}
\addcontentsline{toc}{part}{Частина 2}

\section{Практична робота № 5: Закони розподілу та числові характеристики}
\subsection{Приклади розв'язування задач}

\subsubsection{Приклад 1.1}
Закон розподілу ДВВ (монета).
\textbf{Розв'язання:}
[cite_start]Таблиця: 0 (1/2), 1 (1/2)[cite: 1121].

\subsubsection{Приклад 1.2}
Математичне сподівання.
\textbf{Розв'язання:}
[cite_start]$M(X) = 0 \cdot 0.5 + 1 \cdot 0.5 = 0.5$[cite: 1135].

\subsubsection{Приклад 1.3}
Дисперсія.
\textbf{Розв'язання:}
[cite_start]$D(X) = 0.25$[cite: 1146].

\subsubsection{Приклад 1.4}
Функція розподілу.
\textbf{Розв'язання:}
[cite_start]Східчаста функція зі стрибком 0.5[cite: 1165].

\subsubsection{Приклад 1.5}
Рівномірний розподіл.
\textbf{Розв'язання:}
[cite_start]Функція $F(x) = (x-a)/(b-a)$ на інтервалі[cite: 1187].

\subsubsection{Приклад 1.6}
Нормальний розподіл.
\textbf{Розв'язання:}
[cite_start]Обчислення ймовірності через $\Phi(x)$ для $\mu=3, \sigma=1$[cite: 1231].

\section{Практична робота № 6: Закони розподілу функцій}
\subsection{Приклади розв'язування задач}

\subsubsection{Приклад 2.1}
Закон розподілу $Y=X^3$ для нормального $X$.
\textbf{Розв'язання:}
[cite_start]Знайдено щільність $g(y)$ через похідну оберненої функції[cite: 1345].

\subsubsection{Приклад 2.2}
Сума двох нормальних величин.
\textbf{Розв'язання:}
[cite_start]Використано формулу згортки, доведено, що сума також нормальна[cite: 1357].

% ================= ДОДАНІ РОБОТИ (ПР 3 і 4 з методички ч.2) =================

\section{Практична робота № 7: Найпростіший потік подій. Ланцюги Маркова}
\subsection{Приклади розв'язування задач}

\subsubsection{Приклад 3.1 (Ланцюги Маркова)}
Задано матрицю переходу $P_{1}=\begin{pmatrix} 0.4 & 0.6 \\ 0.3 & 0.7 \end{pmatrix}$.
Знайти матрицю переходу $P_{2}$.

\textbf{Розв'язання:}
Скористаємося формулою $P_{n}=P_{1}^{n}$. Для $n=2$:
$$ P_{2} = P_{1} \cdot P_{1} = \begin{pmatrix} 0.4 & 0.6 \\ 0.3 & 0.7 \end{pmatrix} \cdot \begin{pmatrix} 0.4 & 0.6 \\ 0.3 & 0.7 \end{pmatrix} $$
Обчислимо елементи матриці:
$$ c_{11} = 0.4 \cdot 0.4 + 0.6 \cdot 0.3 = 0.16 + 0.18 = 0.34 $$
$$ c_{12} = 0.4 \cdot 0.6 + 0.6 \cdot 0.7 = 0.24 + 0.42 = 0.66 $$
$$ c_{21} = 0.3 \cdot 0.4 + 0.7 \cdot 0.3 = 0.12 + 0.21 = 0.33 $$
$$ c_{22} = 0.3 \cdot 0.6 + 0.7 \cdot 0.7 = 0.18 + 0.49 = 0.67 $$
Отже:
$$ P_{2} = \begin{pmatrix} 0.34 & 0.66 \\ 0.33 & 0.67 \end{pmatrix} $$
[cite_start][cite: 1590].

\section{Практична робота № 8: Основи вибіркового методу}
\subsection{Приклади розв'язування задач}

\subsubsection{Приклад 4.1 (Вибірка)}
Маса тіла (кг) п'яти навмання вибраних студенток:
$$ X=(50, 65, 55, 55, 60) $$
[cite_start][cite: 1677].

\subsubsection{Приклад 4.2}
Пульс спортсмена протягом п'яти тижнів:
$$ X=(72, 72, 64, 68, 72) $$
[cite_start][cite: 1678].

\subsubsection{Приклад 4.3 (Варіаційний ряд)}
Для вибірки з прикладу 4.1 відсортуємо значення за зростанням:
$$ \alpha=(50, 55, 55, 60, 65) $$
[cite_start][cite: 1686].

\subsubsection{Приклад 4.4 (Статистичний розподіл)}
Для вибірки з прикладу 4.1 складемо таблицю частот:
\begin{center}
\begin{tabular}{|c|c|c|c|c|}
\hline
$X$ & 50 & 55 & 60 & 65 \\
\hline
$\omega$ & 1/5 & 2/5 & 1/5 & 1/5 \\
\hline
\end{tabular}
\end{center}
[cite_start][cite: 1693].

\subsubsection{Приклад 4.5 (Інтервальний ряд)}
Для тієї ж вибірки будуємо інтервали.
\begin{center}
\begin{tabular}{|c|c|c|c|}
\hline
$X$ & $[50, 55)$ & $[55, 60)$ & $[60, 65]$ \\
\hline
$\omega$ & 1/5 & 2/5 & 2/5 \\
\hline
\end{tabular}
\end{center}
[cite_start][cite: 1713].

\subsubsection{Приклад 4.6 (Емпірична функція)}
$$ F_{n}^{*}(x) = \begin{cases}
0, & x \le 50 \\
1/5, & 50 < x \le 55 \\
3/5, & 55 < x \le 60 \\
4/5, & 60 < x \le 65 \\
1, & x > 65
\end{cases} $$
[cite_start][cite: 1724].

\subsubsection{Приклад 4.7 (Медіана)}
Для ряду $\alpha=(50, 55, 55, 60, 65)$ середина — це третій елемент:
$$ \tilde{M}e = 55 $$
Або за формулою середнього двох сусідніх (для прикладу з методички):
[cite_start]$\tilde{M}e = \frac{55+60}{2} = 57.5$[cite: 1735].

\subsubsection{Приклад 4.8 (Середнє арифметичне)}
$$ \overline{x} = \frac{1}{5}(50 \cdot 1 + 55 \cdot 2 + 60 \cdot 1 + 65 \cdot 1) = 57 $$
[cite_start][cite: 1739].

\subsubsection{Приклад 4.9 (Мода)}
Значення, що зустрічається найчастіше (частота 2/5):
$$ Mo = 55 $$
[cite_start][cite: 1749].

\subsubsection{Приклад 4.10 (Розмах)}
$$ R = X_{max} - X_{min} = 65 - 50 = 15 $$
[cite_start][cite: 1753].

\subsubsection{Приклад 4.11 (Дисперсія)}
$$ s^{2} = \frac{1}{5-1} \sum (x_i - 57)^2 = \frac{1}{4} [(-7)^2 + 2(-2)^2 + 3^2 + 8^2] = 32.5 $$
[cite_start][cite: 1756].

\subsubsection{Приклад 4.12 (СКВ)}
$$ s = \sqrt{32.5} \approx 5.701 $$
[cite_start][cite: 1760].

\subsubsection{Приклад 4.13 (MAE)}
Середня абсолютна похибка:
$$ MAE = \frac{1}{5} (|50-57| + |65-57| + \dots) = 4.4 $$
[cite_start][cite: 1764].

\subsubsection{Приклад 4.14 (Асиметрія)}
$$ \tilde{A}_{s} = \frac{n \sum (x_i - \overline{x})^3}{(n-1)(n-2)s^3} \approx 0.405 $$
[cite_start][cite: 1770].

\subsubsection{Приклад 4.15 (Стандартизована асиметрія)}
$$ z_{1} = \frac{0.405}{\sqrt{6/5}} \approx 0.37 $$
[cite_start][cite: 1779].

\subsubsection{Приклад 4.16 (Ексцес)}
$$ \tilde{E}_{k} \approx -0.178 $$
[cite_start][cite: 1783].

\subsubsection{Приклад 4.17 (Стандартизований ексцес)}
$$ z_{2} \approx -0.081 $$
[cite_start][cite: 1792].

\subsubsection{Приклад 4.19 (Довірчий інтервал для середнього)}
Для $\gamma=0.95, n=5$:
$$ 57 - \frac{2.571 \cdot 5.701}{\sqrt{5}} < a < 57 + \frac{2.571 \cdot 5.701}{\sqrt{5}} $$
$$ 50.445 < a < 63.555 $$
[cite_start][cite: 1805].

\subsubsection{Приклад 4.20 (Довірчий інтервал для дисперсії)}
Використовуючи розподіл $\chi^2$:
$$ 11.712 < \sigma^2 < 268.595 $$
Або для СКВ:
$$ 3.422 < \sigma < 16.389 $$
[cite_start][cite: 1811].

\end{document}